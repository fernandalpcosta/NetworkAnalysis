\documentclass[a4paper]{article}

\usepackage{xltxtra}
\usepackage{polyglossia}
\usepackage{fancyhdr}
\usepackage{geometry}

\geometry{a4paper,left=15mm,right=15mm,top=20mm,bottom=20mm}
\pagestyle{fancy}
\lhead{Fernanda Costa}
\chead{Network analysis and visualization with Cytoscape and STRING}
\rhead{\today}
\cfoot{\thepage}

\setlength{\headheight}{23pt}
\setlength{\parindent}{0.0in}
\setlength{\parskip}{0.0in}

\begin{document}

\section*{Exercise \#3}

\vspace{0,5cm}

\subsection*{3.1 Disease query}

Go to \url{https://diseases.jensenlab.org/}, type \textbf{\textit{Parkinson}} in the \textbf{search} field, and click the search button. The web interface will now show the search results, which include all diseases and protein names starting with the search term. Click \textbf{Parkinson’s disease} to get to the results page showing proteins associated with the disease.

\vspace{0,25cm}

Like \textbf{STRING} and \textbf{STITCH}, the \textbf{DISEASES} database integrates several types of evidence, in this case automatic text mining, manually curated knowledge, and experimental evidence from genome-wide association studies.

\begin{itemize}
  \item \textit{Are there any proteins that are supported by all three types of evidence?}
\end{itemize}

\vspace{0,5cm}

\subsection*{3.2 Validation of text mining}
The \textbf{DISEASES} database too allows you to inspect the underlying evidence for an association. Since the predominant source of evidence is automatic text mining, it is always wise to read the underlying text to manually validate the results. Click on \textbf{SNCA} in the \textbf{Text mining table} to view the text based on which it was associated with the disease. Click \textbf{View abstract} for a given entry to see the complete abstract rather than only the title.

\begin{itemize}
  \item \textit{Do the abstracts all mention both the protein and the disease?}
  \item \textit{Do they all use the same name for the protein?}
\end{itemize}

\end{document}
