\documentclass[a4paper]{article}

\usepackage{xltxtra}
\usepackage{polyglossia}
\usepackage{fancyhdr}
\usepackage{geometry}

\geometry{a4paper,left=15mm,right=15mm,top=20mm,bottom=20mm}
\pagestyle{fancy}
\lhead{Fernanda Costa}
\chead{Network analysis and visualization with Cytoscape and STRING}
\rhead{\today}
\cfoot{\thepage}

\setlength{\headheight}{23pt}
\setlength{\parindent}{0.0in}
\setlength{\parskip}{0.0in}

\begin{document}

\section*{Exercise \#1}

\vspace{0,5cm}

\subsection*{1.1 - Single protein query}
We will first retrieve a STRING network for human insulin receptor (INSR). Go to \url{https://string-db.org/}, open the \textbf{Protein by name} search interface, and type \textbf{INSR} in the field \textbf{Protein Name}. You can either specify \textit{Homo sapiens} in the \textbf{Organism} field leave it on auto-detect. Click \textbf{SEARCH}. If you specified the organism, you will immediately receive a protein network; otherwise you will first be presented with a disambiguation page on which you can specify that you meant the human protein.

\vspace{0,5cm}

\subsection*{1.2 Visual representations}
The STRING web interface provides several different visual representations of the network. The \textbf{Settings} tab below the network view allows you to change between different visual representations of the same network. Try changing between the \textbf{confidence} and \textbf{evidence} views; do not forget to press the \textbf{UPDATE} button.
%\vspace{0,25cm}
\begin{itemize}
  \item \textit{Which information is shown for the edges in each representation?}
  \item \textit{Why are there sometimes multiple lines connecting the same two proteins in the evidence representation?}
\end{itemize}

\vspace{0,5cm}

\subsection*{1.3 Evidence viewers}
A key feature of the STRING web interface is the evidence viewers. One should not rely purely on the confidence scores; it is important to inspect the actual evidence underlying an interaction before relying on it, for example, for designing experiments.

\begin{itemize}
  \item \textit{Which types of evidence support the interaction between insulin receptor (INSR) and insulin receptor substrate 1 (IRS1)?}
\end{itemize}

Further detail on the evidence of an interaction can be seen in a popup by clicking on the corresponding edge in the network. Click on the edge between INSR and IRS1 to view its popup; you may need to move the nodes to make this easier.

\begin{itemize}
  \item \textit{Which type of evidence gives the largest contribution to the confidence score?}
\end{itemize}

Click on the \textbf{Show} button to view the experimental evidence for the interaction.

\begin{itemize}
  \item \textit{Which types of experiments support this interaction?}
\end{itemize}

\vspace{0,5cm}

\subsection*{1.4 Query parameters}
The \textbf{Settings} tab also allows you to modify detailed parameters for the search, such as the types of evidence to use (\textbf{active interaction sources}), the \textbf{minimum required interaction score}, and the \textbf{max number of interactors to show}.

\vspace{0,2cm}

Change the minimum required interaction score to high confidence (0.700).

\begin{itemize}
  \item \textit{Does this change the set of proteins shown? Does it change the interactions shown?}
\end{itemize}

Turn off all evidence types except experiments.

\begin{itemize}
  \item \textit{Does this change the set of proteins shown in the network?}
\end{itemize}

Increase the max number of interactors to show to 20.

\begin{itemize}
  \item \textit{How many interaction partners of INSR do you get in the network?}
\end{itemize}

Change the minimum required interaction score back to 0.400.

\begin{itemize}
  \item \textit{How many INSR interactors do you now get?}
\end{itemize}

Turn on all evidence types back on.

\begin{itemize}
  \item \textit{Does this change the set of proteins shown in the network?}
\end{itemize}

\end{document}
